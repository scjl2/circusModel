\documentclass{article}


\usepackage[utf8]{inputenc}
\usepackage{amsmath}
\usepackage{amsfonts}
\usepackage{amssymb}
\usepackage[left=3cm,right=3cm,top=3cm,bottom=3cm]{geometry}
\usepackage{comment}
\usepackage{graphicx}
\usepackage{xcolor}
\usepackage{comment}
\usepackage{soul}
\usepackage{circus}
\usepackage{hijac}
\usepackage{hyperref}


%Content Toggles
\includecomment{comment} %show comments
%\excludecomment{comment}  %do not show comments

%\includecomment{parser} %show comments
\excludecomment{parser}  %do not show comments

\includecomment{plumbing} %show comments
%\excludecomment{plumbing}  %do not show comments

\includecomment{report} %display the title
%\excludecomment{fureportllDoc}  %do not display the title




\title{Safety-Critical Java Level~2 Framework Model}

\author{Matt Luckcuck \\\\Department of Computer Science, \\University of York, UK \\\\ml881@york.ac.uk}

\date{\today}

\begin{document}

\begin{report}
\maketitle

\tableofcontents

\newpage

\section{Introduction}

Safety-Critical Java (SCJ)~\cite{SCJ_Spec_0.100} is a Java-based language for applications that must be certified. To aid certification efforts, SCJ is organised into three compliance levels. Level~0 applications are simple single-processor programs executed by a cyclic executive. By contrast, Level~2 applications are highly concurrent, potentially multi-processor, and make use of suspension and a variety of release patterns.

We model SCJ Level~2 applications using the state-rich process algebra \Circus{}~\cite{circus_woodcock_cavalcanti_2002}. We approach this by splitting our models into a reusable \textit{Framework} model, which captures the API behaviour of SCJ, and a specific \textit{Application} model, which captures the application's behaviour.

Here we present our SCJ Level~2 Framework model, which captures the unchanging behaviour of the API. It is intended that the Framework model be combined with a Application model to produce a model of that specific application. 

\end{report}

\begin{plumbing}
\section{GlobalTypes}
\input{../GlobalTypes.circus}
\newpage

\section{Priority}
\input{../Priority.circus}
\newpage

\section{Priority Queue}
\input{../PriorityQueue.circus}
\newpage

\section{Ids}
\subsection{MissionId}
\input{../MissionId.circus}

\subsection{SchedulableId}
\input{../SchedulableId.circus}
\newpage
\subsection{SchedulableIds}
\input{../SchedulableIds.circus}
\newpage

\section{Channels}
\subsection{FrameworkChan}
\input{../FrameworkChan.circus}

\subsection{SafeletChan}
\input{../SafeletChan.circus}

\subsection{SafeletFWChan}
\input{../SafeletFWChan.circus}

\subsection{SafeletMethChan}
\input{../SafeletMethChan.circus}

\subsection{MissionSequencerMethChan}
\input{../MissionSequencerMethChan.circus}

\subsection{TopLevelMissionSequencerChan}
\input{../TopLevelMissionSequencerChan.circus}

\subsection{TopLevelMissionSequencerFWChan}
\input{../TopLevelMissionSequencerFWChan.circus}

\subsection{MissionChan}
\input{../MissionChan.circus}

\subsection{MissionFWChan}
\input{../MissionChan.circus}

\subsection{MissionMethChan}
\input{../MissionChan.circus}

\subsection{SchedulableChan}
\input{../SchedulableChan.circus}

\subsection{SchedulableMissionSequencerChan}
\input{../SchedulableMissionSequencerChan.circus}

\subsection{SchedulableMissionSequencerFWChan}
\input{../SchedulableMissionSequencerFWChan.circus}

\subsection{ManagedThreadChan}
\input{../ManagedThreadChan.circus}

\subsection{ManagedThreadFWChan}
\input{../ManagedThreadFWChan.circus}

\subsection{ManagedThreadMethChan}
\input{../ManagedThreadMethChan.circus}
\newpage
\end{plumbing}

\section{ObjectFW}
\input{../ObjectFW.circus}
\newpage

\section{ThreadFW}
\input{../ThreadFW.circus}
\newpage

\section{SafeletFW}
\input{../SafeletFW.circus}
\newpage

\section{TopLevelMissionSequencerFW}
\input{../TopLevelMissionSequencerFW.circus}
\newpage

\section{MissionFW}
\input{../MissionFW.circus}
\newpage

\section{SchedulableMissionSequencerFW}
\input{../SchedulableMissionSequencerFW.circus}
\newpage

\section{Event Handlers}
\subsection{AperiodicEventHandlerFW}
\input{../AperiodicEventHandlerFW.circus}
\newpage

\subsection{PeriodicEventHandlerFW}
\input{../PeriodicEventHandlerFW.circus}
\newpage

\subsection{OneShotEventHandlerFW}
\input{../OneShotEventHandlerFW.circus}
\newpage

\section{ManagedThreadFW}
\input{../ManagedThreadFW.circus}
\newpage

\bibliographystyle{plain}
\bibliography{FrameworkBib}

\end{document}
